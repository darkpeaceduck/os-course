\begin{frame}
\frametitle{x86 рушит наши планы}
\begin{itemize}
  \item<1-> В x86 есть несколько чувствительных непривилегированных инструкций
    \begin{itemize}
      \item например, инструкция \emph{SGDT} позволяет получить указатель на
            \emph{GDT};
      \item код ОС может считывать и проверять значение указателя \emph{GDT};
      \item код ОС может считать оттуда совсем не то, что она туда положила;
    \end{itemize}
  \item<2-> x86 довольно популярная платформа
    \begin{itemize}
      \item виртуализуемая или нет - мне надо!
      \item в новых версиях x86 появилась поддержка эффективной виртуализации;
      \item но и до появления этой поддержки компании VMWare удавалось довольно
            эффективно виртуализовывать x86.
    \end{itemize}
\end{itemize}
\end{frame}

\begin{frame}
\frametitle{Бинарная трансляция}
\begin{itemize}
  \item<1-> Вернемся к старой и медленной симуляции
    \begin{itemize}
      \item зная, что \emph{guest} и \emph{host} используют одну архитектуру
            мы можем довольно быстро "переписать" код;
      \item нам нужно найти в коде "опасные" инструкции и заменить их на вызовы
            гиппервизора;
    \end{itemize}
  \item<2-> Как найти опасные инструкции?
    \begin{itemize}
      \item Как вообще найти код среди всей памяти в системе?
      \item Что является кодом, что является данными определяется в момент
            исполнения.
      \item Мы знает точку входа, далее мы можем отслеживать куда передается
            управление.
    \end{itemize}
\end{itemize}
\end{frame}

\begin{frame}
\frametitle{Бинарная трансляция}
\begin{itemize}
  \item Транслятор работает с \emph{базовыми блоками}:
    \begin{itemize}
      \item базовый блок - линейная последовательность инструкций заканчивающаяся
            инструкцией перехода;
      \item трансляция базового блока - замена "опасных" инструкций на вызов
            \emph{VMM} и, возможно, вставка какого-то эпилога;
      \item для каждого транслированного блока поддерживается его адрес и размер
            в \emph{VM};
      \item транслированные блоки можно кешировать;
      \item транслировать код, выполняющийся в непривилегированном режиме,
            обычно, не нужно;
    \end{itemize}
\end{itemize}
\end{frame}

\begin{frame}[fragile]
\frametitle{Пример}
\begin{columns}[T]
  \begin{column}{.4\linewidth}
    \begin{lstlisting}
    addl  $4,         %esp
    movl  %cr0,       %eax
    orl   $(1 << 31), %eax
    movl  %eax, %cr0
    pushl $enable_64bit_gdt
    call  videomem_puts
    \end{lstlisting}
  \end{column}
  \begin{column}{.4\linewidth}
    \begin{lstlisting}
    addl  $4,         %esp
    call  trap_get_cr0
    orl   $(1 << 31), %eax
    pushl %eax
    call  trap_set_cr0
    pushl $enable_64bit_gdt
    pushl $videomem_puts
    call  back_to_vmm
    \end{lstlisting}
  \end{column}
\end{columns}
\begin{itemize}
  \item В примере \emph{VMM} перехватывает операции с регистром \emph{\%cr0};
  \item Инструкция перехода в конце транслируется в вызов \emph{VMM}
\end{itemize}
\end{frame}

\begin{frame}
\frametitle{Аппаратная поддержка виртуализации}
\begin{itemize}
  \item Intel-VT и AMD-V - расширения архитектуры для поддержки эффективной
        виртуализации
    \begin{itemize}
      \item детали различаются, но обе технологии позволяют перехватывать
            чувствительные инструкции;
      \item первое поколение Intel-VT и AMD-V проигрывало по скорости бинарной
            трансляции из-за большой стоимости передачи управления между
            \emph{VMM} и ОС;
      \item т. е. эффективная вирутализация оказалась не такой эффективной;
    \end{itemize}
\end{itemize}
\end{frame}
